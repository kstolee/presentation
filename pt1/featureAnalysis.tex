\subsection{Feature Analysis}

{
\setbeamertemplate{background canvas}{\tikz[remember picture]\node[opacity=0.5] at (current page.center) {\includegraphics[keepaspectratio]{nontex/illustrations/galtonBoard.jpg}};}
\begin{frame}
\frametitle{What Features Are Used Most Frequently?}
\begin{block}{\begin{Large}Regex Feature Usage References Are Missing\end{Large}}
\begin{itemize}
\item \begin{large}feature usage statistics\end{large}
\item \begin{large}feature set summaries for variants and tools\end{large}
\end{itemize}
\end{block}
\end{frame}
}
\note[itemize]{
    \item https://41.media.tumblr.com/deca33c79b8c4ae87b6d5cd1ffe25155/tumblr_inline_nmak5hmd9T1sh0dgh_540.jpg
    \item What features does each language and tool support...
\item \emph{so I know how to port my regex-containing code?}
\item \emph{so I can choose the best fitting regex variant for my needs?}
\item \emph{so I can choose the best analysis tool?}
\item What features are more frequently used, and thus more important when...
\item \emph{building an analysis tool?}
\item \emph{developing test regexes?}
\item \emph{creating a toy regex language for a formal experiment?}
}


%------------------------------------------------


\begin{frame}[fragile]
\frametitle{Why Python?}
\begin{footnotesize}Python has most commonly shared features, few rarely shared features.\end{footnotesize}
\begin{columns}[t] % The "c" option specifies centered vertical alignment while the "t" option is used for top vertical alignment
\column{.45\textwidth} % Left column and width
\begin{adjustbox}{totalheight=\textheight-4\baselineskip}
\begin{tabular}{ll@{  }cc@{  }lc @{   } c @{   }c @{   }c @{   }c @{   }c @{   }c @{   }c}\textbf{Code} & \textbf{Example} & \textbf{Python} & \textbf{Perl} & \textbf{.Net}  & \textbf{Ruby} &  \textbf{Java} & \textbf{RE2} & \textbf{JavaScript}& \textbf{POSIX ERE}\\
\toprule
ADD & \begin{minipage}{0.5in}\begin{verbatim}z+\end{verbatim}\end{minipage} & \yes & \yes & \yes & \yes & \yes & \yes & \yes & \yes\\
\midrule
CG & \begin{minipage}{0.5in}\begin{verbatim}(caught)\end{verbatim}\end{minipage} & \yes & \yes & \yes & \yes & \yes & \yes & \yes & \yes\\
\midrule
KLE & \begin{minipage}{0.5in}\begin{verbatim}.*\end{verbatim}\end{minipage} & \yes & \yes & \yes & \yes & \yes & \yes & \yes & \yes\\
\midrule
CCC & \begin{minipage}{0.5in}\begin{verbatim}[aeiou]\end{verbatim}\end{minipage} & \yes & \yes & \yes & \yes & \yes & \yes & \yes & \yes\\
\midrule
ANY & \begin{minipage}{0.5in}\begin{verbatim}.\end{verbatim}\end{minipage} & \yes & \yes & \yes & \yes & \yes & \yes & \yes & \yes\\
\midrule
RNG & \begin{minipage}{0.5in}\begin{verbatim}[a-z]\end{verbatim}\end{minipage} & \yes & \yes & \yes & \yes & \yes & \yes & \yes & \yes\\
\midrule
STR & \begin{minipage}{0.5in}\begin{verbatim}^\end{verbatim}\end{minipage} & \yes & \yes & \yes & \yes & \yes & \yes & \yes & \yes\\
\midrule
END & \begin{minipage}{0.5in}\begin{verbatim}$\end{verbatim}\end{minipage} & \yes & \yes & \yes & \yes & \yes & \yes & \yes & \yes\\
\midrule[0.10em]
NCCC & \begin{minipage}{0.5in}\begin{verbatim}[^qwxf]\end{verbatim}\end{minipage} & \yes & \yes & \yes & \yes & \yes & \yes & \yes & \yes\\
\midrule
WSP & \begin{minipage}{0.5in}\begin{verbatim}\s\end{verbatim}\end{minipage} & \yes & \yes & \yes & \yes & \yes & \yes & \yes & \no\\
\midrule
OR & \begin{minipage}{0.5in}\begin{verbatim}a|b\end{verbatim}\end{minipage} & \yes & \yes & \yes & \yes & \yes & \yes & \yes & \yes\\
\midrule
DEC & \begin{minipage}{0.5in}\begin{verbatim}\d\end{verbatim}\end{minipage} & \yes & \yes & \yes & \yes & \yes & \yes & \yes & \no\\
\midrule
WRD & \begin{minipage}{0.5in}\begin{verbatim}\w\end{verbatim}\end{minipage} & \yes & \yes & \yes & \yes & \yes & \yes & \yes & \no\\
\midrule
QST & \begin{minipage}{0.5in}\begin{verbatim}z?\end{verbatim}\end{minipage} & \yes & \yes & \yes & \yes & \yes & \yes & \yes & \yes\\
\midrule
LZY & \begin{minipage}{0.5in}\begin{verbatim}z+?\end{verbatim}\end{minipage} & \yes & \yes & \yes & \yes & \yes & \yes & \yes & \no\\
\midrule
NCG & \begin{minipage}{0.5in}\begin{verbatim}a(?:b)c\end{verbatim}\end{minipage} & \yes & \yes & \yes & \yes & \yes & \yes & \yes & \no\\
\midrule
PNG & \begin{minipage}{0.5in}\begin{verbatim}(?P<name>x)\end{verbatim}\end{minipage} & \yes & \yes & \no & \no & \no & \yes & \no & \no\\
\midrule
SNG & \begin{minipage}{0.5in}\begin{verbatim}z{8}\end{verbatim}\end{minipage} & \yes & \yes & \yes & \yes & \yes & \yes & \yes & \yes\\
\midrule
NWSP & \begin{minipage}{0.5in}\begin{verbatim}\S\end{verbatim}\end{minipage} & \yes & \yes & \yes & \yes & \yes & \yes & \yes & \no\\
\midrule
DBB & \begin{minipage}{0.5in}\begin{verbatim}z{3,8}\end{verbatim}\end{minipage} & \yes & \yes & \yes & \yes & \yes & \yes & \yes & \yes\\
\midrule
NLKA & \begin{minipage}{0.5in}\begin{verbatim}a(?!yz)\end{verbatim}\end{minipage} & \yes & \yes & \yes & \yes & \yes & \no & \yes & \no\\
\midrule
WNW & \begin{minipage}{0.5in}\begin{verbatim}\b\end{verbatim}\end{minipage} & \yes & \yes & \yes & \yes & \yes & \yes & \yes & \no\\
\midrule
NWRD & \begin{minipage}{0.5in}\begin{verbatim}\W\end{verbatim}\end{minipage} & \yes & \yes & \yes & \yes & \yes & \yes & \yes & \no\\
\midrule
LWB & \begin{minipage}{0.5in}\begin{verbatim}z{15,}\end{verbatim}\end{minipage} & \yes & \yes & \yes & \yes & \yes & \yes & \yes & \yes\\
\midrule
LKA & \begin{minipage}{0.5in}\begin{verbatim}a(?=bc)\end{verbatim}\end{minipage} & \yes & \yes & \yes & \yes & \yes & \no & \yes & \no\\
\midrule
OPT & \begin{minipage}{0.5in}\begin{verbatim}(?i)CasE\end{verbatim}\end{minipage} & \yes & \yes & \yes & \yes & \yes & \yes & \no & \no\\
\midrule
NLKB & \begin{minipage}{0.5in}\begin{verbatim}(?<!x)yz\end{verbatim}\end{minipage} & \yes & \yes & \yes & \yes & \yes & \no & \no & \no\\
\midrule
LKB & \begin{minipage}{0.5in}\begin{verbatim}(?<=a)bc\end{verbatim}\end{minipage} & \yes & \yes & \yes & \yes & \yes & \no & \no & \no\\
\midrule
ENDZ & \begin{minipage}{0.5in}\begin{verbatim}\Z\end{verbatim}\end{minipage} & \yes & \no & \no & \no & \no & \no & \no & \no\\
\midrule
BKR & \begin{minipage}{0.5in}\begin{verbatim}\1\end{verbatim}\end{minipage} & \yes & \yes & \yes & \yes & \yes & \no & \yes & \yes\\
\midrule
NDEC & \begin{minipage}{0.5in}\begin{verbatim}\D\end{verbatim}\end{minipage} & \yes & \yes & \yes & \yes & \yes & \yes & \yes & \no\\
\midrule
BKRN & \begin{minipage}{0.5in}\begin{verbatim}(P?=name)\end{verbatim}\end{minipage} & \yes & \yes & \no & \no & \no & \no & \no & \no\\
\midrule
VWSP & \begin{minipage}{0.5in}\begin{verbatim}\v\end{verbatim}\end{minipage} & \yes & \yes & \yes & \no & \yes & \yes & \yes & \yes\\
\midrule
NWNW & \begin{minipage}{0.5in}\begin{verbatim}\B\end{verbatim}\end{minipage} & \yes & \yes & \yes & \yes & \yes & \yes & \yes & \no\\
\bottomrule
\end{tabular}
\end{adjustbox}

\column{.45\textwidth} % Right column and width
\input{table/features/unrankedFeatureSupport}
\end{columns}


\end{frame}
\note[itemize]{
    \item pt 1
    \item pt 2
}

%------------------------------------------------

\begin{frame}
\frametitle{Project Selection}
\begin{columns}[t] % The "c" option specifies centered vertical alignment while the "t" option is used for top vertical alignment
\column{.5\textwidth} % Left column and width
\begin{figure}[ht]
  \includegraphics[scale=0.16]{nontex/illustrations/32Divided.eps}
  \label{fig:32Divided}
\end{figure}
\column{.5\textwidth} % Right column and width
\begin{figure}[ht]
  \animategraphics[loop,width=\linewidth]{10}{nontex/smallScrapersPNG/smallScrapers-}{0}{400}
    \label{fig:scraper}
\end{figure}
Out of 3,898 pseudo-randomly selected Python projects, 1,645 contained one or more utilization.
\end{columns}
\end{frame}
\note[itemize]{
    \item in each section, average 1312 examined
    \item in each section, average 122 .py found
}

%------------------------------------------------

\begin{frame}
\frametitle{Utilizations of the re module}
\begin{figure}[h]
  \centering
  \includegraphics[scale=0.77]{nontex/illustrations/exampleUsageLarge.eps}
  \label{fig:exampleUsageLarge}
\end{figure}
\begin{description}
\item [function] which function of the re module is called?
\item [pattern] string used to specify regex behavior
\item [flags] modifies the regex engine
\end{description}
\end{frame}
\note[itemize]{
    \item 53,894 unique utilizations observed.
    \item pt 2
}

%------------------------------------------------


\begin{frame}
\frametitle{re module insights}
\begin{figure}[ht]
  \includegraphics[scale=0.6]{nontex/illustrations/partFunctions.eps}
  \label{fig:partFunctions}
\end{figure}
\begin{figure}[ht]
  \includegraphics[scale=0.6]{nontex/illustrations/partFlags.eps}
  \label{fig:partFlags}
\end{figure}
\end{frame}
\note[itemize]{
    \item average utilizations per project was 32 and the maximum was 1,427
    \item each project had an average of 11 files containing any utilization, but med was 6, so there is a skew bc of a few projects with so many
    \item each of these files had an average of 2 utilizations (med 1)
    \item avg 2 util. per file
    \item max 207 util. per file
    \item max 541 files w/utilizations, med was 2, Q3 was 6...this got skewed
    \item
}

%------------------------------------------------
{
\setbeamertemplate{background canvas}{\tikz[remember picture,overlay]\node[opacity=0.3] at (current page.center) {\includegraphics[width=\paperwidth,height=\paperheight,keepaspectratio]{nontex/illustrations/funnel.png}};}
\begin{frame}
\frametitle{Filtering Utilizations And Patterns}
\begin{itemize}
\item [] \textbf{53,894} unique utilizations observed.
\item [] \begin{footnotesize}12.7\% use behavioral flags\end{footnotesize}
\item [] \begin{footnotesize}6.5\% were non-static patterns\end{footnotesize}
\item [] \textbf{43,525} utilizations remain
\item [] \textbf{13,711} distinct normalized patterns
\item [] \begin{footnotesize}73 had unsupported Unicode characters\end{footnotesize}
\item [] \begin{footnotesize}17 had non-Python features\end{footnotesize}
\item [] \begin{footnotesize}22 had various errors\end{footnotesize}
\item [] \begin{footnotesize}2 had ECOM feature - too rare to include\end{footnotesize}
\item [] \textbf{13,597} usable patterns remain for analysis
\end{itemize}
\end{frame}
}
\note[itemize]{
    \item https://pixabay.com/static/uploads/photo/2014/04/03/10/03/funnel-309721_960_720.png
    \item pt 2
}

%------------------------------------------------



\begin{frame}
\frametitle{PCRE Parsing Patterns}
\begin{figure}[h]
  \centering
  \includegraphics[scale=1]{nontex/illustrations/featureParsing.eps}
  \label{fig:featureParsing}
\end{figure}
\begin{center}
\begin{Large}
All Python features are recognizable by PCRE
\end{Large}
\end{center}
\end{frame}
\note[itemize]{
    \item pt 1
    \item pt 2
}

%------------------------------------------------

\begin{frame}[fragile]
\frametitle{Feature Statistics}
\begin{columns}[t]
\column{.47\textwidth}
\begin{adjustbox}{totalheight=\textheight-9\baselineskip}
\begin{tabular}
{lllcccc  cc}
\textbf{Rank} & \textbf{Code} & \textbf{Example} & \% \textbf{Projects} & \textbf{NProjects} & \textbf{NFiles} & \textbf{NPatterns} & \textbf{MaxTokens} \\
\toprule[0.12em]
1 & ADD & \begin{minipage}{0.5in}\begin{verbatim}z+\end{verbatim}\end{minipage} & 73.2 & 1,204 & 9,165 & 6,003 & 30 \\
\midrule
2 & CG & \begin{minipage}{0.5in}\begin{verbatim}(caught)\end{verbatim}\end{minipage} & 72.6 & 1,194 & 9,559 & 7,130 & 17 \\
\midrule
3 & KLE & \begin{minipage}{0.5in}\begin{verbatim}.*\end{verbatim}\end{minipage} & 66.8 & 1,099 & 8,163 & 6,017 & 50 \\
\midrule
4 & CCC & \begin{minipage}{0.5in}\begin{verbatim}[aeiou]\end{verbatim}\end{minipage} & 62.4 & 1,026 & 7,648 & 4,468 & 42 \\
\midrule
5 & ANY & \begin{minipage}{0.5in}\begin{verbatim}.\end{verbatim}\end{minipage} & 61.1 & 1,005 & 6,277 & 4,657 & 60 \\
\midrule
6 & RNG & \begin{minipage}{0.5in}\begin{verbatim}[a-z]\end{verbatim}\end{minipage} & 51.6 & 848 & 5,092 & 2,631 & 50 \\
\midrule
7 & STR & \begin{minipage}{0.5in}\begin{verbatim}^\end{verbatim}\end{minipage} & 51.4 & 846 & 5,458 & 3,563 & 12 \\
\midrule
8 & END & \begin{minipage}{0.5in}\begin{verbatim}$\end{verbatim}\end{minipage} & 50.3 & 827 & 5,393 & 3,169 & 12 \\
\midrule[0.10em]
9 & NCCC & \begin{minipage}{0.5in}\begin{verbatim}[^qwxf]\end{verbatim}\end{minipage} & 47.2 & 776 & 3,947 & 1,935 & 15 \\
\midrule
10 & WSP & \begin{minipage}{0.5in}\begin{verbatim}\s\end{verbatim}\end{minipage} & 46.3 & 762 & 4,704 & 2,846 & 32 \\
\midrule
11 & OR & \begin{minipage}{0.5in}\begin{verbatim}a|b\end{verbatim}\end{minipage} & 43 & 708 & 3,926 & 2,102 & 15 \\
\midrule
12 & DEC & \begin{minipage}{0.5in}\begin{verbatim}\d\end{verbatim}\end{minipage} & 42.1 & 692 & 4,198 & 2,297 & 24 \\
\midrule
13 & WRD & \begin{minipage}{0.5in}\begin{verbatim}\w\end{verbatim}\end{minipage} & 39.5 & 650 & 2,952 & 1,430 & 13 \\
\midrule
14 & QST & \begin{minipage}{0.5in}\begin{verbatim}z?\end{verbatim}\end{minipage} & 39.2 & 645 & 3,707 & 1,871 & 35 \\
\midrule
15 & LZY & \begin{minipage}{0.5in}\begin{verbatim}z+?\end{verbatim}\end{minipage} & 36.8 & 605 & 2,221 & 1,300 & 12 \\
\midrule
16 & NCG & \begin{minipage}{0.5in}\begin{verbatim}a(?:b)c\end{verbatim}\end{minipage} & 24.6 & 404 & 1,709 & 791 & 28 \\
\midrule
17 & PNG & \begin{minipage}{0.5in}\begin{verbatim}(?P<name>x)\end{verbatim}\end{minipage} & 21.5 & 354 & 1,475 & 915 & 16 \\
\bottomrule[0.13em]
\end{tabular}
\end{adjustbox}
\column{.47\textwidth}
\begin{adjustbox}{totalheight=\textheight-9\baselineskip}
\begin{tabular}
{lllcccc  cc}
\textbf{Rank} & \textbf{Code} & \textbf{Example} & \% \textbf{Projects} & \textbf{NProjects} & \textbf{NFiles} & \textbf{NPatterns} & \textbf{MaxTokens} \\
\toprule[0.12em]
18 & SNG & \begin{minipage}{0.5in}\begin{verbatim}z{8}\end{verbatim}\end{minipage} & 20.7 & 340 & 1,267 & 581 & 17 \\
\midrule
19 & NWSP & \begin{minipage}{0.5in}\begin{verbatim}\S\end{verbatim}\end{minipage} & 16.4 & 270 & 776 & 484 & 10 \\
\midrule
20 & DBB & \begin{minipage}{0.5in}\begin{verbatim}z{3,8}\end{verbatim}\end{minipage} & 14.5 & 238 & 647 & 367 & 11 \\
\midrule
21 & NLKA & \begin{minipage}{0.5in}\begin{verbatim}a(?!yz)\end{verbatim}\end{minipage} & 11.1 & 183 & 489 & 131 & 3 \\
\midrule
22 & WNW & \begin{minipage}{0.5in}\begin{verbatim}\b\end{verbatim}\end{minipage} & 10.1 & 166 & 438 & 248 & 36 \\
\midrule
23 & NWRD & \begin{minipage}{0.5in}\begin{verbatim}\W\end{verbatim}\end{minipage} & 10 & 165 & 305 & 94 & 6 \\
\midrule
24 & LWB & \begin{minipage}{0.5in}\begin{verbatim}z{15,}\end{verbatim}\end{minipage} & 9.6 & 158 & 281 & 91 & 3 \\
\midrule
25 & LKA & \begin{minipage}{0.5in}\begin{verbatim}a(?=bc)\end{verbatim}\end{minipage} & 9.6 & 158 & 358 & 112 & 4 \\
\midrule
26 & OPT & \begin{minipage}{0.5in}\begin{verbatim}(?i)CasE\end{verbatim}\end{minipage} & 9.4 & 154 & 377 & 231 & 2 \\
\midrule
27 & NLKB & \begin{minipage}{0.5in}\begin{verbatim}(?<!x)yz\end{verbatim}\end{minipage} & 8.3 & 137 & 296 & 94 & 4 \\
\midrule
28 & LKB & \begin{minipage}{0.5in}\begin{verbatim}(?<=a)bc\end{verbatim}\end{minipage} & 7.3 & 120 & 255 & 80 & 4 \\
\midrule
29 & ENDZ & \begin{minipage}{0.5in}\begin{verbatim}\Z\end{verbatim}\end{minipage} & 5.5 & 90 & 149 & 89 & 1 \\
\midrule
30 & BKR & \begin{minipage}{0.5in}\begin{verbatim}\1\end{verbatim}\end{minipage} & 5.1 & 84 & 129 & 60 & 4 \\
\midrule
31 & NDEC & \begin{minipage}{0.5in}\begin{verbatim}\D\end{verbatim}\end{minipage} & 3.5 & 58 & 92 & 36 & 6 \\
\midrule
32 & BKRN & \begin{minipage}{0.5in}\begin{verbatim}(P?=name)\end{verbatim}\end{minipage} & 1.7 & 28 & 44 & 17 & 2 \\
\midrule
33 & VWSP & \begin{minipage}{0.5in}\begin{verbatim}\v\end{verbatim}\end{minipage} & 0.9 & 15 & 16 & 13 & 2 \\
\midrule
34 & NWNW & \begin{minipage}{0.5in}\begin{verbatim}\B\end{verbatim}\end{minipage} & 0.7 & 11 & 11 & 4 & 2 \\
\bottomrule[0.13em]
\end{tabular}
\end{adjustbox}
\end{columns}

\end{frame}
\note[itemize]{
    \item pt 1
    \item pt 2
}

%------------------------------------------------

\begin{frame}[fragile]
\frametitle{Feature Statistics - Top 8}
\begin{adjustbox}{width=\textwidth}
\begin{tabular}
{lllcccc  cc}
\textbf{Rank} & \textbf{Code} & \textbf{Example} & \% \textbf{Projects} & \textbf{NProjects} & \textbf{NFiles} & \textbf{NPatterns} & \textbf{MaxTokens} \\
\toprule[0.12em]
1 & ADD & \begin{minipage}{0.5in}\begin{verbatim}z+\end{verbatim}\end{minipage} & 73.2 & 1,204 & 9,165 & 6,003 & 30 \\
\midrule
2 & CG & \begin{minipage}{0.5in}\begin{verbatim}(caught)\end{verbatim}\end{minipage} & 72.6 & 1,194 & 9,559 & 7,130 & 17 \\
\midrule
3 & KLE & \begin{minipage}{0.5in}\begin{verbatim}.*\end{verbatim}\end{minipage} & 66.8 & 1,099 & 8,163 & 6,017 & 50 \\
\midrule
4 & CCC & \begin{minipage}{0.5in}\begin{verbatim}[aeiou]\end{verbatim}\end{minipage} & 62.4 & 1,026 & 7,648 & 4,468 & 42 \\
\midrule
5 & ANY & \begin{minipage}{0.5in}\begin{verbatim}.\end{verbatim}\end{minipage} & 61.1 & 1,005 & 6,277 & 4,657 & 60 \\
\midrule
6 & RNG & \begin{minipage}{0.5in}\begin{verbatim}[a-z]\end{verbatim}\end{minipage} & 51.6 & 848 & 5,092 & 2,631 & 50 \\
\midrule
7 & STR & \begin{minipage}{0.5in}\begin{verbatim}^\end{verbatim}\end{minipage} & 51.4 & 846 & 5,458 & 3,563 & 12 \\
\midrule
8 & END & \begin{minipage}{0.5in}\begin{verbatim}$\end{verbatim}\end{minipage} & 50.3 & 827 & 5,393 & 3,169 & 12 \\
\bottomrule[0.13em]
\end{tabular}
\end{adjustbox}

\end{frame}
\note[itemize]{
    \item present in more than 50\% of projects
    \item pt 2
}

%------------------------------------------------



\begin{frame}[fragile]
\frametitle{Ranked features: Languages}
\input{table/features/splitLanguages}
\end{frame}
\note[itemize]{
    \item Hampi supports the most features (25 features), followed by Rex (21 features), Automata.Z3 (14 features) and brics (12 features).
}

%------------------------------------------------

\begin{frame}[fragile]
\frametitle{Ranked features: Languages - Notable Missing Features}
\begin{adjustbox}{width=\textwidth}
\begin{tabular}{l@{  }clc@{  }lc @{   } c @{   }c @{   }c @{   }c @{   }c @{   }c @{   }c}\textbf{Rank} & \textbf{Code} & \textbf{Example} & \textbf{Python} & \textbf{Perl} & \textbf{.Net}  & \textbf{Ruby} &  \textbf{Java} & \textbf{RE2} & \textbf{JavaScript} & \textbf{POSIX ERE}\\
\toprule
21 & NLKA & \begin{minipage}{0.5in}\begin{verbatim}a(?!yz)\end{verbatim}\end{minipage} & \yes & \yes & \yes & \yes & \yes & \eek & \yes & \eek\\
\midrule
22 & WNW & \begin{minipage}{0.5in}\begin{verbatim}\b\end{verbatim}\end{minipage} & \yes & \yes & \yes & \yes & \yes & \yes & \yes & \eek\\
\midrule
23 & NWRD & \begin{minipage}{0.5in}\begin{verbatim}\W\end{verbatim}\end{minipage} & \yes & \yes & \yes & \yes & \yes & \yes & \yes & \eek\\
\midrule
24 & LWB & \begin{minipage}{0.5in}\begin{verbatim}z{15,}\end{verbatim}\end{minipage} & \yes & \yes & \yes & \yes & \yes & \yes & \yes & \yes\\
\midrule
25 & LKA & \begin{minipage}{0.5in}\begin{verbatim}a(?=bc)\end{verbatim}\end{minipage} & \yes & \yes & \yes & \yes & \yes & \eek & \yes & \eek\\
\midrule
26 & OPT & \begin{minipage}{0.5in}\begin{verbatim}(?i)CasE\end{verbatim}\end{minipage} & \yes & \yes & \yes & \yes & \yes & \yes & \eek & \eek\\
\midrule
27 & NLKB & \begin{minipage}{0.5in}\begin{verbatim}(?<!x)yz\end{verbatim}\end{minipage} & \yes & \yes & \yes & \yes & \yes & \eek & \eek & \eek\\
\midrule
28 & LKB & \begin{minipage}{0.5in}\begin{verbatim}(?<=a)bc\end{verbatim}\end{minipage} & \yes & \yes & \yes & \yes & \yes & \eek & \eek & \eek\\
\midrule
29 & ENDZ & \begin{minipage}{0.5in}\begin{verbatim}\Z\end{verbatim}\end{minipage} & \yes & \no & \no & \no & \no & \no & \no & \no\\
\midrule
30 & BKR & \begin{minipage}{0.5in}\begin{verbatim}\1\end{verbatim}\end{minipage} & \yes & \yes & \yes & \yes & \yes & \eek & \yes & \yes\\
\midrule
31 & NDEC & \begin{minipage}{0.5in}\begin{verbatim}\D\end{verbatim}\end{minipage} & \yes & \yes & \yes & \yes & \yes & \yes & \yes & \eek\\
\bottomrule
\end{tabular}
\end{adjustbox}

\end{frame}
\note[itemize]{
    \item pt 1
    \item pt 2
}

%------------------------------------------------


\begin{frame}
\frametitle{Ranked features: Analysis Tools}
\input{table/features/splitTools}
\end{frame}
\note[itemize]{
    \item pt 1
    \item pt 2
}

%------------------------------------------------


